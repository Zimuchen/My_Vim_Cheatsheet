% Created by Zmchary, as of 2016-12-31
% modified 2017-01-31
% Fell free to circulate this cheatsheet.
\documentclass[10pt,landscape]{article}
\usepackage{multicol}
\usepackage{calc}
\usepackage{ifthen}
\usepackage[landscape]{geometry}
\usepackage{amsmath,amsthm,amsfonts,amssymb}
\usepackage[usenames, dvipsnames]{color}
\usepackage{graphicx,overpic}
\usepackage{soul}
\usepackage{hyperref}



\pdfinfo{
  /Creator (LaTeX)
  /Producer (pdfTeX 1.40.0)
  /Author (Zimu)
  /Subject (Cheatsheet)
  /Keywords (Vim, latex, pdftex, tex)}

% This sets page margins to .5 inch if using letter paper, and to 1cm
% if using A4 paper. (This probably isn't strictly necessary.)
% If using another size paper, use default 1cm margins.
\ifthenelse{\lengthtest { \paperwidth = 11in}}
    { \geometry{top=.2in,left=.2in,right=.2in,bottom=.2in} }
    {\ifthenelse{ \lengthtest{ \paperwidth = 297mm}}
        {\geometry{top=1cm,left=1cm,right=1cm,bottom=1cm} }
        {\geometry{top=1cm,left=1cm,right=1cm,bottom=1cm} }
    }

% Turn off header and footer
\pagestyle{empty}
\pagenumbering{gobble}

% Redefine section commands to use less space
\makeatletter
\renewcommand{\section}{\@startsection{section}{1}{0mm}%
                                {-1ex plus -.5ex minus -.2ex}%
                                {0.5ex plus .2ex}%x
                                {\normalfont\large\bfseries}}
\renewcommand{\subsection}{\@startsection{subsection}{2}{0mm}%
                                {-1ex plus -.5ex minus -.2ex}%
                                {0.5ex plus .2ex}%
                                {\normalfont\normalsize\bfseries}}
\renewcommand{\subsubsection}{\@startsection{subsubsection}{3}{0mm}%
                                {-1ex plus -.5ex minus -.2ex}%
                                {1ex plus .2ex}%
                                {\normalfont\small\bfseries}}
\makeatother

% Don't print section numbers
\setcounter{secnumdepth}{0}


\setlength{\parindent}{0pt}
\setlength{\parskip}{0pt plus 0.5ex}



% \author{Zmchary}
% \title{Cheatsheet for Vim}
\renewcommand{\title}[1]{{\normalfont\sffamily\huge\bfseries #1}\vspace{1ex}}
\def\leader{<Leader>}
\def\locleader{<LocalLeader>}
\def\S{\textcolor{red}{\tt search }}

\begin{document}
\raggedright
\title{Cheatsheet for Vim}
\hskip 3cm {\tt \leader}: {\tt ,} \hskip 1cm {\tt \locleader}: {\textbf{\textbackslash}}%
\newlength{\MyLen}
\settowidth{\MyLen}{\texttt{letterpaper}/\texttt{a4paper} \ }


\begin{multicols}{3}
\section{Common Use}
\begin{tabular}{@{}p{\the\MyLen}@{}p{\linewidth-\the\MyLen}@{}}
    \hline \hline
   Default shortcut   & Menu entry       \\
   \hline
   \verb|zq|          & :close          \\
   \verb|zQ|          & :qa!            \\
   \verb|zs|          & :update         \\
   \hline
   \verb|<F2>|        & Toggle VimFiler       \\
   \verb|<F3>|        & show Tagbar           \\
   \verb|<F8>|        & change colour           \\
   \verb|<F9>|        & paste toggle           \\
   \hline
   \verb|:map|        & show all key mapping           \\
   \verb|<Leader> <Enter>| & show all key mapping by fzf \\
   \hline
   \verb|<Leader> z|  & like winredo   \\
   \verb|<Leader> c|  & close Quickfix \\
   \hline
   \verb|<Leader> G|  & Goyo           \\
    \hline \hline
   % \verb|t|           & add a new line below           \\
   % \verb|T|           & add a new line above           \\
   % \hline
   % \verb|<leader> t=| & Tabularize with \verb|=|           \\
   % \verb|<leader> t-| & Tabularize with \verb|<-|          \\
   % \hline
   % \verb|<leader> ve| & edit .vimrc \\
   % \verb|<leader> vs| & source .vimrc\\
   % \hline
   % \verb|<C-p>|       & CtrlP           \\
   % \verb|<leader> b|  & :CtrlPBuffer    \\
   % \verb|<F12>|       & show number           \\
   % \verb|<Leader> x|  & Toggle auto close Mappings           \\
\end{tabular}
\vfill

\section{Jumps}
\begin{tabular}{@{}p{\the\MyLen}@{}p{\linewidth-\the\MyLen}@{}}
  \hline \hline
   Default shortcut   & Menu entry       \\
  \hline
    Quickfix          &                 \\
   \verb|]q|          & :cnext          \\
   \verb|[q|          & :cprev          \\
   \verb|]l|          & :lnext          \\
   \verb|[l|          & :lprev          \\
   \verb|<Leader> c|  & close          \\
  \hline
     Buffers/Tabs     &                 \\
   \verb|]b|          & :bnext          \\
   \verb|[b|          & :bprev          \\
   \verb|]t|          & :tabn           \\
   \verb|[t|          & :tabp           \\
  \hline
     Tab              &                 \\
   \verb|<tab>|       & next windown    \\
   \verb|<S-tab>|     & previous windown\\
  \hline
    Line            &                          \\ 
  \verb|<C-o>|      & moves to the last jump   \\
  \verb|<C-<TAB>>|  & back to the jump   \\
  \verb|<C-i>|      & ??   \\
   \hline\hline
\end{tabular}
\vfill
\section{show mark} 
\begin{tabular}{@{}p{\the\MyLen}@{}p{\linewidth-\the\MyLen}@{}}
    \hline \hline
   Default shortcut & Menu entry       \\
  \hline
  \verb|mx |        & Toggle mark 'x' \\
  \verb|dmx|        & Remove mark 'x' \\
  \hline
  \verb|'x|         & jump to mark 'x' \\
  \verb|'.|         & jump to the last edit made \\
  \hline
  \verb|:mark|      & shows all marks set \\
  \verb|:jumps|     & shows jump list \\
  \hline \hline
\end{tabular}
\vfill

\section{fold} 
\begin{tabular}{@{}p{\the\MyLen}@{}p{\linewidth-\the\MyLen}@{}}
  \hline \hline
   Default shortcut & Menu entry       \\
  \hline
   \verb|zi|        & focue model          \\
   \verb|zo|        & open this fold          \\
   \verb|zc|        & smart close           \\
   \verb|zj|        & move to next fold           \\
   \verb|zk|        & move to previous fold           \\
   \verb|zr|        & open \textcolor{ForestGreen}{All} fold (nested)   \\
   \verb|zm(M)|     & close \textcolor{ForestGreen}{All} fold       \\
   % \verb|zf|        & create a fold           \\
   \hline
   \verb|z[|        & put foldmarker \verb|{{{| \\
   \verb|z]|        & put foldmarker \verb|}}}| \\
  \hline \hline
\end{tabular}
\vfill

\section{fzf}
\begin{tabular}{@{}p{\the\MyLen}@{}p{\linewidth-\the\MyLen}@{}}
  \hline \hline
   Default shortcut     & Menu entry       \\
  \hline
   \verb|<Leader> bb|   & show all buffers           \\
   \verb|<Leader> fw|   & \S word in all files         \\
   \verb|<Leader> fl|   & \S under word in all files     \\
   \verb|<Leader> mm|   & \S in all marks           \\
  \hline
    \verb|<C-X><C-l>|     & complete-line    \\
    \verb|<C-X><C-j>|     & complete-file-name-ag    \\
    \verb|<Leader><Enter>|   & fzf-maps           \\
  \hline \hline
\end{tabular}
\vfill

\section{unite}
\begin{tabular}{@{}p{\the\MyLen}@{}p{\linewidth-\the\MyLen}@{}}
  \hline \hline
  Default shortcut       & Menu entry       \\
  \hline
   \verb|<C-p>|         & open file in current fold    \\
  \verb|[unite]|         & \verb|,|           \\
  \verb|[unite] me|      & show message           \\
  \verb|[unite] mr|       & show register        \\
   % \verb|[unite] bb|       & search all files           \\
   %\verb|[unite] r|       & search in register           \\
  \verb|[unite] mo|       & show outline           \\
  \verb|[unite] fs|       & \S anything in this file       \\
  \hline \hline
\end{tabular}
\vfill

\section{easymotion} 
\begin{tabular}{@{}p{\the\MyLen}@{}p{\linewidth-\the\MyLen}@{}}
    \hline \hline
   Default shortcut  & Menu entry       \\
  \hline
   easymotion-prefix & \verb|,,|         \\
   \verb|,,s|        & Find \{char\}           \\
   \verb|,,f|        & find \{char\} to the right      \\
   \verb|,,F|        & find \{char\} to the left       \\
   \verb|,,w(W)|     & Beginning of the word (forward)      \\
   \verb|,,b(B)|     & Beginning of the word (back)         \\
   \verb|,,e(E)|     & End of the word (forward)         \\
   \verb|,,ge|       & End of the word (back)   \\
   \verb|,,n|        & Jump to lastest ``/''           \\
    \hline \hline
\end{tabular}
\vfill

\section{vim-surround} 
\begin{tabular}{@{}p{\the\MyLen}@{}p{\linewidth-\the\MyLen}@{}}
  \hline \hline
   Default shortcut      & Menu entry       \\
  \hline
   \verb|ds{"}|          & delete           \\
   \verb|cs])|           & change {\tt[]} to {\tt()}    \\
   \verb|ysw(W){"}|      & add        \\
   \textit{v}\verb|S{'}| & \textit{v} mode           \\
  \hline \hline
\end{tabular}
\vfill

\section{text-objects} 
\verb|<S-v>| to start {\tt text-object} model ({\tt x} model)
\begin{tabular}{@{}p{\the\MyLen}@{}p{\linewidth-\the\MyLen}@{}}
  \hline \hline
   Default shortcut      & Menu entry       \\
   \hline
   \verb|+|           & expand region           \\
   \verb|-|           & shrinkage region        \\
   \hline
   \verb|i(a)i|           & indent object        \\
   \verb|i(a)_|           & between the char {\tt \_{}}  \\
   \verb|i(a)l|           & line object without indent       \\
  \hline \hline
\end{tabular}
\vfill

\section{Incsearch}
\begin{tabular}{@{}p{\the\MyLen}@{}p{\linewidth-\the\MyLen}@{}}
  \hline \hline
   Default shortcut       & Menu entry       \\
  \hline
   \verb|/|               & \verb|;|           \\
   \verb|?/|              & \S word in buffer           \\
   \verb|g/|              & \S all files           \\
   \verb|n(N)|            & \S in register           \\
   \verb|*(#)|            & \S in register           \\
   \verb|g*(#)|           & \S in register           \\
  \hline
   \verb|:%s/patten/bar/g|&  {\tt \%}: range all lines          \\
   \verb|:%s/patten/bar/gc|& {\tt g}: global in a line      \\
   \verb|:%s/patten/bar/gc|& {\tt c}: ask for confirmation      \\
   \verb|:5,12s/patten/bar/g|&  range line 5 to 12          \\
   \verb|:'a,'b/patten/bar/g|&  range mark {\tt a} to mark {\tt b}  \\
   \url{http://vim.wikia.com/wiki/Search_and_replace}\\
  \hline \hline
\end{tabular}
\vfill

\section{Command}
\begin{tabular}{@{}p{\the\MyLen}@{}p{\linewidth-\the\MyLen}@{}}
  \hline \hline
   Default shortcut       & Menu entry       \\
   \hline
   \verb|:Color|          & change colour theme           \\
   \verb|:Plug<TAB>|      & plug install etc.           \\
   \verb|:Chomp|          & clean space            \\
  \hline \hline
\end{tabular}
\vfill
\section{Nvim-R}
\begin{tabular}{@{}p{\the\MyLen}@{}p{\linewidth-\the\MyLen}@{}}
    \hline \hline
   Default shortcut  & Menu entry       \\
  \hline
   \verb|<LocalLeader> rf|       & Start R (default)\\
   \verb|<LocalLeader> rq|       & Close R (no save)\\
   \hline
   \verb|<LocalLeader> ff|       & Function (cur)                       \\
   \verb|<LocalLeader> fe|       & Function (cur, echo)                 \\
   \verb|<LocalLeader> fd|       & Function (cur and down)              \\
   \verb|<LocalLeader> fa|       & Function (cur, echo and down)        \\
   \hline
   \verb|<LocalLeader> l|        & Line                                  \\
   \verb|<LocalLeader> d|        & Line (and down)                       \\
   \verb|<LocalLeader> q|        & Line (and new one)                    \\
   % \verb|<LocalLeader> r<Left>|  & Left part of line (cur)        \\
   % \verb|<LocalLeader> r<Right>| & Right part of line (cur)       \\
   % \verb|<LocalLeader> o|        & Line (evaluate and insert the output as comment)  \\
   \hline
   \verb|<LocalLeader> ro|       & Show/Update Object Browser        \\
   \verb|<LocalLeader> r=|       & Expand (all lists)                \\
   \hline
   \verb|<LocalLeader> rl|       & List space             \\
   \verb|<LocalLeader> rr|       & Clear console          \\
   \verb|<LocalLeader> rm|       & Clear all              \\
   \verb|<LocalLeader> rp|       & Print (cur)            \\
   \verb|<LocalLeader> rn|       & Names (cur)            \\
   \verb|<LocalLeader> rt|       & Structure (cur)        \\
   \verb|<LocalLeader> rv|       & View data.frame (cur)  \\
   \hline
   \verb|<C-x><C-o>| & Complete object name              \\
   \verb|<C-d>|      & Complete object name without \$   \\
   \verb|<C-x><C-a>| & Complete function arguments       \\
   \verb|==|         & Indent (line)                     \\
   \verb|=|          & Indent (selected lines)           \\
   \verb|g=G|        & Indent (whole buffer)             \\
    \hline \hline
\end{tabular}
\vfill
% section  (end)

\section{Fugitive}
\begin{tabular}{@{}p{\the\MyLen}@{}p{\linewidth-\the\MyLen}@{}}
  \hline \hline
   Default shortcut       & Menu entry       \\
   \hline
   \verb|:Gwrite|          & stage           \\
   \verb|:Gread|          & revert to last checked           \\
   \verb|:Gremove|      & remove a file           \\
   \verb|:Gmove|          & rename            \\
  \hline \hline
\end{tabular}
\vfill

\newpage

\section{Shell}
\begin{tabular}{@{}p{\the\MyLen}@{}p{\linewidth-\the\MyLen}@{}}
  \hline \hline
   Default shortcut       & Menu entry       \\
   \hline
   \verb|<C-r>|          & \S search command in history  \\
   \verb|ag|             & search everything           \\
  \hline \hline
\end{tabular}
\vfill


\end{multicols}

\end{document}
